\section{Solutions techniques}
	Pour la réalisation de ce projet, nous avons choisi d'utiliser le langage Java. De plus, pour la gestion des diagrammes, nous avons également utilisé une librairie dédiée à cette tâche : JFreeChart. De façon à structurer les sources, nous avons mis en place une structure MVC (Model-View-Controler). Cette façon de programmer permet de séparer les différents modules en les rendant les plus indépendants possibles les uns des autres. Cela améliore la maintenabilité ainsi que la réutilisabilité du code.\\
	En suivant cette méthode, voici comment nous avons organisé nos sources :
	
	\begin{enumerate}
		\item Package "Controlleur"
		\begin{itemize}
			\item \textbf{Controller} : organise les interactions entre les classes du package Modele elle-même mais aussi entre les classes du package Modele et du package Vue
		\end{itemize}
		
		\item Package "Modele"
		\begin{itemize}
			\item \textbf{Grille} : chargée de gérer la génération de la grille et de ses obstacles
			\item \textbf{MathModel} : implémentation du modèle mathématique : fourni la prochaine position d'une personne
			\item \textbf{Neighborhood} : fourni les cases voisines d'une personne
			\item \textbf{Obstacle} : représente un obstacle
			\item \textbf{Person} : représente une personne
		\end{itemize}
		
		\item Package "Vue"
		\begin{itemize}
			\item \textbf{Chart} : dessine un diagramme en batons
			\item \textbf{ChartLine} : dessine un diagramme dessinant une courbe
			\item \textbf{DrawAreaUI} : aire de dessin des composants de l'application
			\item \textbf{GrilleUI} : représente la partie graphique d'une instance de Grille 
			\item \textbf{MainWindow} : fenêtre principale (contient la fonction main())
			\item \textbf{PersonUI} : représente la partie graphique d'une instance de Personne
		\end{itemize}
	\end{enumerate}	
	
	Le schéma UML suivant fourni une vision globale des classes implémentées : ~\ref{fig:uml}\\
	
%	% Graphic for TeX using PGF
% Title: /home/albin/Developpements/EclipseWorkspace/Automaton-Cellular/UML.dia
% Creator: Dia v0.97.2
% CreationDate: Fri May 18 21:42:13 2012
% For: albin
% \usepackage{tikz}
% The following commands are not supported in PSTricks at present
% We define them conditionally, so when they are implemented,
% this pgf file will use them.
\ifx\du\undefined
  \newlength{\du}
\fi
\setlength{\du}{15\unitlength}
\begin{tikzpicture}
\pgftransformxscale{1.000000}
\pgftransformyscale{-1.000000}
\definecolor{dialinecolor}{rgb}{0.000000, 0.000000, 0.000000}
\pgfsetstrokecolor{dialinecolor}
\definecolor{dialinecolor}{rgb}{1.000000, 1.000000, 1.000000}
\pgfsetfillcolor{dialinecolor}
\pgfsetlinewidth{0.100000\du}
\pgfsetdash{}{0pt}
\definecolor{dialinecolor}{rgb}{1.000000, 1.000000, 1.000000}
\pgfsetfillcolor{dialinecolor}
\fill (1.400000\du,1.900000\du)--(1.400000\du,10.800000\du)--(14.450000\du,10.800000\du)--(14.450000\du,1.900000\du)--cycle;
\definecolor{dialinecolor}{rgb}{0.000000, 0.000000, 0.000000}
\pgfsetstrokecolor{dialinecolor}
\draw (1.400000\du,1.900000\du)--(1.400000\du,10.800000\du)--(14.450000\du,10.800000\du)--(14.450000\du,1.900000\du)--cycle;
\definecolor{dialinecolor}{rgb}{1.000000, 1.000000, 1.000000}
\pgfsetfillcolor{dialinecolor}
\fill (1.400000\du,0.900000\du)--(1.400000\du,1.900000\du)--(3.525000\du,1.900000\du)--(3.525000\du,0.900000\du)--cycle;
\definecolor{dialinecolor}{rgb}{0.000000, 0.000000, 0.000000}
\pgfsetstrokecolor{dialinecolor}
\draw (1.400000\du,0.900000\du)--(1.400000\du,1.900000\du)--(3.525000\du,1.900000\du)--(3.525000\du,0.900000\du)--cycle;
% setfont left to latex
\definecolor{dialinecolor}{rgb}{0.000000, 0.000000, 0.000000}
\pgfsetstrokecolor{dialinecolor}
\node[anchor=west] at (1.500000\du,1.650000\du){Model};
\pgfsetlinewidth{0.100000\du}
\pgfsetdash{}{0pt}
\definecolor{dialinecolor}{rgb}{1.000000, 1.000000, 1.000000}
\pgfsetfillcolor{dialinecolor}
\fill (1.750000\du,2.400000\du)--(1.750000\du,3.800000\du)--(7.272500\du,3.800000\du)--(7.272500\du,2.400000\du)--cycle;
\definecolor{dialinecolor}{rgb}{0.000000, 0.000000, 0.000000}
\pgfsetstrokecolor{dialinecolor}
\draw (1.750000\du,2.400000\du)--(1.750000\du,3.800000\du)--(7.272500\du,3.800000\du)--(7.272500\du,2.400000\du)--cycle;
% setfont left to latex
\definecolor{dialinecolor}{rgb}{0.000000, 0.000000, 0.000000}
\pgfsetstrokecolor{dialinecolor}
\node at (4.511250\du,3.350000\du){MathModel};
\definecolor{dialinecolor}{rgb}{1.000000, 1.000000, 1.000000}
\pgfsetfillcolor{dialinecolor}
\fill (1.750000\du,3.800000\du)--(1.750000\du,4.200000\du)--(7.272500\du,4.200000\du)--(7.272500\du,3.800000\du)--cycle;
\definecolor{dialinecolor}{rgb}{0.000000, 0.000000, 0.000000}
\pgfsetstrokecolor{dialinecolor}
\draw (1.750000\du,3.800000\du)--(1.750000\du,4.200000\du)--(7.272500\du,4.200000\du)--(7.272500\du,3.800000\du)--cycle;
\definecolor{dialinecolor}{rgb}{1.000000, 1.000000, 1.000000}
\pgfsetfillcolor{dialinecolor}
\fill (1.750000\du,4.200000\du)--(1.750000\du,4.600000\du)--(7.272500\du,4.600000\du)--(7.272500\du,4.200000\du)--cycle;
\definecolor{dialinecolor}{rgb}{0.000000, 0.000000, 0.000000}
\pgfsetstrokecolor{dialinecolor}
\draw (1.750000\du,4.200000\du)--(1.750000\du,4.600000\du)--(7.272500\du,4.600000\du)--(7.272500\du,4.200000\du)--cycle;
\pgfsetlinewidth{0.100000\du}
\pgfsetdash{}{0pt}
\definecolor{dialinecolor}{rgb}{1.000000, 1.000000, 1.000000}
\pgfsetfillcolor{dialinecolor}
\fill (1.780000\du,5.025000\du)--(1.780000\du,6.425000\du)--(8.670000\du,6.425000\du)--(8.670000\du,5.025000\du)--cycle;
\definecolor{dialinecolor}{rgb}{0.000000, 0.000000, 0.000000}
\pgfsetstrokecolor{dialinecolor}
\draw (1.780000\du,5.025000\du)--(1.780000\du,6.425000\du)--(8.670000\du,6.425000\du)--(8.670000\du,5.025000\du)--cycle;
% setfont left to latex
\definecolor{dialinecolor}{rgb}{0.000000, 0.000000, 0.000000}
\pgfsetstrokecolor{dialinecolor}
\node at (5.225000\du,5.975000\du){Neighborhood};
\definecolor{dialinecolor}{rgb}{1.000000, 1.000000, 1.000000}
\pgfsetfillcolor{dialinecolor}
\fill (1.780000\du,6.425000\du)--(1.780000\du,6.825000\du)--(8.670000\du,6.825000\du)--(8.670000\du,6.425000\du)--cycle;
\definecolor{dialinecolor}{rgb}{0.000000, 0.000000, 0.000000}
\pgfsetstrokecolor{dialinecolor}
\draw (1.780000\du,6.425000\du)--(1.780000\du,6.825000\du)--(8.670000\du,6.825000\du)--(8.670000\du,6.425000\du)--cycle;
\definecolor{dialinecolor}{rgb}{1.000000, 1.000000, 1.000000}
\pgfsetfillcolor{dialinecolor}
\fill (1.780000\du,6.825000\du)--(1.780000\du,7.225000\du)--(8.670000\du,7.225000\du)--(8.670000\du,6.825000\du)--cycle;
\definecolor{dialinecolor}{rgb}{0.000000, 0.000000, 0.000000}
\pgfsetstrokecolor{dialinecolor}
\draw (1.780000\du,6.825000\du)--(1.780000\du,7.225000\du)--(8.670000\du,7.225000\du)--(8.670000\du,6.825000\du)--cycle;
\pgfsetlinewidth{0.100000\du}
\pgfsetdash{}{0pt}
\definecolor{dialinecolor}{rgb}{1.000000, 1.000000, 1.000000}
\pgfsetfillcolor{dialinecolor}
\fill (11.160000\du,7.850000\du)--(11.160000\du,9.250000\du)--(13.560000\du,9.250000\du)--(13.560000\du,7.850000\du)--cycle;
\definecolor{dialinecolor}{rgb}{0.000000, 0.000000, 0.000000}
\pgfsetstrokecolor{dialinecolor}
\draw (11.160000\du,7.850000\du)--(11.160000\du,9.250000\du)--(13.560000\du,9.250000\du)--(13.560000\du,7.850000\du)--cycle;
% setfont left to latex
\definecolor{dialinecolor}{rgb}{0.000000, 0.000000, 0.000000}
\pgfsetstrokecolor{dialinecolor}
\node at (12.360000\du,8.800000\du){Grid};
\definecolor{dialinecolor}{rgb}{1.000000, 1.000000, 1.000000}
\pgfsetfillcolor{dialinecolor}
\fill (11.160000\du,9.250000\du)--(11.160000\du,9.650000\du)--(13.560000\du,9.650000\du)--(13.560000\du,9.250000\du)--cycle;
\definecolor{dialinecolor}{rgb}{0.000000, 0.000000, 0.000000}
\pgfsetstrokecolor{dialinecolor}
\draw (11.160000\du,9.250000\du)--(11.160000\du,9.650000\du)--(13.560000\du,9.650000\du)--(13.560000\du,9.250000\du)--cycle;
\definecolor{dialinecolor}{rgb}{1.000000, 1.000000, 1.000000}
\pgfsetfillcolor{dialinecolor}
\fill (11.160000\du,9.650000\du)--(11.160000\du,10.050000\du)--(13.560000\du,10.050000\du)--(13.560000\du,9.650000\du)--cycle;
\definecolor{dialinecolor}{rgb}{0.000000, 0.000000, 0.000000}
\pgfsetstrokecolor{dialinecolor}
\draw (11.160000\du,9.650000\du)--(11.160000\du,10.050000\du)--(13.560000\du,10.050000\du)--(13.560000\du,9.650000\du)--cycle;
\pgfsetlinewidth{0.100000\du}
\pgfsetdash{}{0pt}
\definecolor{dialinecolor}{rgb}{1.000000, 1.000000, 1.000000}
\pgfsetfillcolor{dialinecolor}
\fill (1.790000\du,7.725000\du)--(1.790000\du,9.125000\du)--(6.232500\du,9.125000\du)--(6.232500\du,7.725000\du)--cycle;
\definecolor{dialinecolor}{rgb}{0.000000, 0.000000, 0.000000}
\pgfsetstrokecolor{dialinecolor}
\draw (1.790000\du,7.725000\du)--(1.790000\du,9.125000\du)--(6.232500\du,9.125000\du)--(6.232500\du,7.725000\du)--cycle;
% setfont left to latex
\definecolor{dialinecolor}{rgb}{0.000000, 0.000000, 0.000000}
\pgfsetstrokecolor{dialinecolor}
\node at (4.011250\du,8.675000\du){Obstacle};
\definecolor{dialinecolor}{rgb}{1.000000, 1.000000, 1.000000}
\pgfsetfillcolor{dialinecolor}
\fill (1.790000\du,9.125000\du)--(1.790000\du,9.525000\du)--(6.232500\du,9.525000\du)--(6.232500\du,9.125000\du)--cycle;
\definecolor{dialinecolor}{rgb}{0.000000, 0.000000, 0.000000}
\pgfsetstrokecolor{dialinecolor}
\draw (1.790000\du,9.125000\du)--(1.790000\du,9.525000\du)--(6.232500\du,9.525000\du)--(6.232500\du,9.125000\du)--cycle;
\definecolor{dialinecolor}{rgb}{1.000000, 1.000000, 1.000000}
\pgfsetfillcolor{dialinecolor}
\fill (1.790000\du,9.525000\du)--(1.790000\du,9.925000\du)--(6.232500\du,9.925000\du)--(6.232500\du,9.525000\du)--cycle;
\definecolor{dialinecolor}{rgb}{0.000000, 0.000000, 0.000000}
\pgfsetstrokecolor{dialinecolor}
\draw (1.790000\du,9.525000\du)--(1.790000\du,9.925000\du)--(6.232500\du,9.925000\du)--(6.232500\du,9.525000\du)--cycle;
\pgfsetlinewidth{0.100000\du}
\pgfsetdash{}{0pt}
\definecolor{dialinecolor}{rgb}{1.000000, 1.000000, 1.000000}
\pgfsetfillcolor{dialinecolor}
\fill (9.970000\du,2.350000\du)--(9.970000\du,3.750000\du)--(13.590000\du,3.750000\du)--(13.590000\du,2.350000\du)--cycle;
\definecolor{dialinecolor}{rgb}{0.000000, 0.000000, 0.000000}
\pgfsetstrokecolor{dialinecolor}
\draw (9.970000\du,2.350000\du)--(9.970000\du,3.750000\du)--(13.590000\du,3.750000\du)--(13.590000\du,2.350000\du)--cycle;
% setfont left to latex
\definecolor{dialinecolor}{rgb}{0.000000, 0.000000, 0.000000}
\pgfsetstrokecolor{dialinecolor}
\node at (11.780000\du,3.300000\du){Person};
\definecolor{dialinecolor}{rgb}{1.000000, 1.000000, 1.000000}
\pgfsetfillcolor{dialinecolor}
\fill (9.970000\du,3.750000\du)--(9.970000\du,4.150000\du)--(13.590000\du,4.150000\du)--(13.590000\du,3.750000\du)--cycle;
\definecolor{dialinecolor}{rgb}{0.000000, 0.000000, 0.000000}
\pgfsetstrokecolor{dialinecolor}
\draw (9.970000\du,3.750000\du)--(9.970000\du,4.150000\du)--(13.590000\du,4.150000\du)--(13.590000\du,3.750000\du)--cycle;
\definecolor{dialinecolor}{rgb}{1.000000, 1.000000, 1.000000}
\pgfsetfillcolor{dialinecolor}
\fill (9.970000\du,4.150000\du)--(9.970000\du,4.550000\du)--(13.590000\du,4.550000\du)--(13.590000\du,4.150000\du)--cycle;
\definecolor{dialinecolor}{rgb}{0.000000, 0.000000, 0.000000}
\pgfsetstrokecolor{dialinecolor}
\draw (9.970000\du,4.150000\du)--(9.970000\du,4.550000\du)--(13.590000\du,4.550000\du)--(13.590000\du,4.150000\du)--cycle;
\pgfsetlinewidth{0.100000\du}
\pgfsetdash{}{0pt}
\definecolor{dialinecolor}{rgb}{1.000000, 1.000000, 1.000000}
\pgfsetfillcolor{dialinecolor}
\fill (1.450000\du,12.650000\du)--(1.450000\du,16.150000\du)--(9.700000\du,16.150000\du)--(9.700000\du,12.650000\du)--cycle;
\definecolor{dialinecolor}{rgb}{0.000000, 0.000000, 0.000000}
\pgfsetstrokecolor{dialinecolor}
\draw (1.450000\du,12.650000\du)--(1.450000\du,16.150000\du)--(9.700000\du,16.150000\du)--(9.700000\du,12.650000\du)--cycle;
\definecolor{dialinecolor}{rgb}{1.000000, 1.000000, 1.000000}
\pgfsetfillcolor{dialinecolor}
\fill (1.450000\du,11.650000\du)--(1.450000\du,12.650000\du)--(5.500000\du,12.650000\du)--(5.500000\du,11.650000\du)--cycle;
\definecolor{dialinecolor}{rgb}{0.000000, 0.000000, 0.000000}
\pgfsetstrokecolor{dialinecolor}
\draw (1.450000\du,11.650000\du)--(1.450000\du,12.650000\du)--(5.500000\du,12.650000\du)--(5.500000\du,11.650000\du)--cycle;
% setfont left to latex
\definecolor{dialinecolor}{rgb}{0.000000, 0.000000, 0.000000}
\pgfsetstrokecolor{dialinecolor}
\node[anchor=west] at (1.550000\du,12.400000\du){Controller};
\pgfsetlinewidth{0.100000\du}
\pgfsetdash{}{0pt}
\definecolor{dialinecolor}{rgb}{1.000000, 1.000000, 1.000000}
\pgfsetfillcolor{dialinecolor}
\fill (3.100000\du,13.350000\du)--(3.100000\du,14.750000\du)--(8.122500\du,14.750000\du)--(8.122500\du,13.350000\du)--cycle;
\definecolor{dialinecolor}{rgb}{0.000000, 0.000000, 0.000000}
\pgfsetstrokecolor{dialinecolor}
\draw (3.100000\du,13.350000\du)--(3.100000\du,14.750000\du)--(8.122500\du,14.750000\du)--(8.122500\du,13.350000\du)--cycle;
% setfont left to latex
\definecolor{dialinecolor}{rgb}{0.000000, 0.000000, 0.000000}
\pgfsetstrokecolor{dialinecolor}
\node at (5.611250\du,14.300000\du){Controller};
\definecolor{dialinecolor}{rgb}{1.000000, 1.000000, 1.000000}
\pgfsetfillcolor{dialinecolor}
\fill (3.100000\du,14.750000\du)--(3.100000\du,15.150000\du)--(8.122500\du,15.150000\du)--(8.122500\du,14.750000\du)--cycle;
\definecolor{dialinecolor}{rgb}{0.000000, 0.000000, 0.000000}
\pgfsetstrokecolor{dialinecolor}
\draw (3.100000\du,14.750000\du)--(3.100000\du,15.150000\du)--(8.122500\du,15.150000\du)--(8.122500\du,14.750000\du)--cycle;
\definecolor{dialinecolor}{rgb}{1.000000, 1.000000, 1.000000}
\pgfsetfillcolor{dialinecolor}
\fill (3.100000\du,15.150000\du)--(3.100000\du,15.550000\du)--(8.122500\du,15.550000\du)--(8.122500\du,15.150000\du)--cycle;
\definecolor{dialinecolor}{rgb}{0.000000, 0.000000, 0.000000}
\pgfsetstrokecolor{dialinecolor}
\draw (3.100000\du,15.150000\du)--(3.100000\du,15.550000\du)--(8.122500\du,15.550000\du)--(8.122500\du,15.150000\du)--cycle;
\pgfsetlinewidth{0.100000\du}
\pgfsetdash{}{0pt}
\definecolor{dialinecolor}{rgb}{1.000000, 1.000000, 1.000000}
\pgfsetfillcolor{dialinecolor}
\fill (15.350000\du,1.750000\du)--(15.350000\du,10.850000\du)--(31.600000\du,10.850000\du)--(31.600000\du,1.750000\du)--cycle;
\definecolor{dialinecolor}{rgb}{0.000000, 0.000000, 0.000000}
\pgfsetstrokecolor{dialinecolor}
\draw (15.350000\du,1.750000\du)--(15.350000\du,10.850000\du)--(31.600000\du,10.850000\du)--(31.600000\du,1.750000\du)--cycle;
\definecolor{dialinecolor}{rgb}{1.000000, 1.000000, 1.000000}
\pgfsetfillcolor{dialinecolor}
\fill (15.350000\du,0.750000\du)--(15.350000\du,1.750000\du)--(17.350000\du,1.750000\du)--(17.350000\du,0.750000\du)--cycle;
\definecolor{dialinecolor}{rgb}{0.000000, 0.000000, 0.000000}
\pgfsetstrokecolor{dialinecolor}
\draw (15.350000\du,0.750000\du)--(15.350000\du,1.750000\du)--(17.350000\du,1.750000\du)--(17.350000\du,0.750000\du)--cycle;
% setfont left to latex
\definecolor{dialinecolor}{rgb}{0.000000, 0.000000, 0.000000}
\pgfsetstrokecolor{dialinecolor}
\node[anchor=west] at (15.450000\du,1.500000\du){View};
\pgfsetlinewidth{0.100000\du}
\pgfsetdash{}{0pt}
\definecolor{dialinecolor}{rgb}{1.000000, 1.000000, 1.000000}
\pgfsetfillcolor{dialinecolor}
\fill (15.950000\du,7.800000\du)--(15.950000\du,9.200000\du)--(19.297500\du,9.200000\du)--(19.297500\du,7.800000\du)--cycle;
\definecolor{dialinecolor}{rgb}{0.000000, 0.000000, 0.000000}
\pgfsetstrokecolor{dialinecolor}
\draw (15.950000\du,7.800000\du)--(15.950000\du,9.200000\du)--(19.297500\du,9.200000\du)--(19.297500\du,7.800000\du)--cycle;
% setfont left to latex
\definecolor{dialinecolor}{rgb}{0.000000, 0.000000, 0.000000}
\pgfsetstrokecolor{dialinecolor}
\node at (17.623750\du,8.750000\du){GridUI};
\definecolor{dialinecolor}{rgb}{1.000000, 1.000000, 1.000000}
\pgfsetfillcolor{dialinecolor}
\fill (15.950000\du,9.200000\du)--(15.950000\du,9.600000\du)--(19.297500\du,9.600000\du)--(19.297500\du,9.200000\du)--cycle;
\definecolor{dialinecolor}{rgb}{0.000000, 0.000000, 0.000000}
\pgfsetstrokecolor{dialinecolor}
\draw (15.950000\du,9.200000\du)--(15.950000\du,9.600000\du)--(19.297500\du,9.600000\du)--(19.297500\du,9.200000\du)--cycle;
\definecolor{dialinecolor}{rgb}{1.000000, 1.000000, 1.000000}
\pgfsetfillcolor{dialinecolor}
\fill (15.950000\du,9.600000\du)--(15.950000\du,10.000000\du)--(19.297500\du,10.000000\du)--(19.297500\du,9.600000\du)--cycle;
\definecolor{dialinecolor}{rgb}{0.000000, 0.000000, 0.000000}
\pgfsetstrokecolor{dialinecolor}
\draw (15.950000\du,9.600000\du)--(15.950000\du,10.000000\du)--(19.297500\du,10.000000\du)--(19.297500\du,9.600000\du)--cycle;
\pgfsetlinewidth{0.100000\du}
\pgfsetdash{}{0pt}
\definecolor{dialinecolor}{rgb}{1.000000, 1.000000, 1.000000}
\pgfsetfillcolor{dialinecolor}
\fill (21.780000\du,8.125000\du)--(21.780000\du,9.525000\du)--(24.755000\du,9.525000\du)--(24.755000\du,8.125000\du)--cycle;
\definecolor{dialinecolor}{rgb}{0.000000, 0.000000, 0.000000}
\pgfsetstrokecolor{dialinecolor}
\draw (21.780000\du,8.125000\du)--(21.780000\du,9.525000\du)--(24.755000\du,9.525000\du)--(24.755000\du,8.125000\du)--cycle;
% setfont left to latex
\definecolor{dialinecolor}{rgb}{0.000000, 0.000000, 0.000000}
\pgfsetstrokecolor{dialinecolor}
\node at (23.267500\du,9.075000\du){Chart};
\definecolor{dialinecolor}{rgb}{1.000000, 1.000000, 1.000000}
\pgfsetfillcolor{dialinecolor}
\fill (21.780000\du,9.525000\du)--(21.780000\du,9.925000\du)--(24.755000\du,9.925000\du)--(24.755000\du,9.525000\du)--cycle;
\definecolor{dialinecolor}{rgb}{0.000000, 0.000000, 0.000000}
\pgfsetstrokecolor{dialinecolor}
\draw (21.780000\du,9.525000\du)--(21.780000\du,9.925000\du)--(24.755000\du,9.925000\du)--(24.755000\du,9.525000\du)--cycle;
\definecolor{dialinecolor}{rgb}{1.000000, 1.000000, 1.000000}
\pgfsetfillcolor{dialinecolor}
\fill (21.780000\du,9.925000\du)--(21.780000\du,10.325000\du)--(24.755000\du,10.325000\du)--(24.755000\du,9.925000\du)--cycle;
\definecolor{dialinecolor}{rgb}{0.000000, 0.000000, 0.000000}
\pgfsetstrokecolor{dialinecolor}
\draw (21.780000\du,9.925000\du)--(21.780000\du,10.325000\du)--(24.755000\du,10.325000\du)--(24.755000\du,9.925000\du)--cycle;
\pgfsetlinewidth{0.100000\du}
\pgfsetdash{}{0pt}
\definecolor{dialinecolor}{rgb}{1.000000, 1.000000, 1.000000}
\pgfsetfillcolor{dialinecolor}
\fill (24.960000\du,5.100000\du)--(24.960000\du,6.500000\du)--(30.845000\du,6.500000\du)--(30.845000\du,5.100000\du)--cycle;
\definecolor{dialinecolor}{rgb}{0.000000, 0.000000, 0.000000}
\pgfsetstrokecolor{dialinecolor}
\draw (24.960000\du,5.100000\du)--(24.960000\du,6.500000\du)--(30.845000\du,6.500000\du)--(30.845000\du,5.100000\du)--cycle;
% setfont left to latex
\definecolor{dialinecolor}{rgb}{0.000000, 0.000000, 0.000000}
\pgfsetstrokecolor{dialinecolor}
\node at (27.902500\du,6.050000\du){DrawAreaUI};
\definecolor{dialinecolor}{rgb}{1.000000, 1.000000, 1.000000}
\pgfsetfillcolor{dialinecolor}
\fill (24.960000\du,6.500000\du)--(24.960000\du,6.900000\du)--(30.845000\du,6.900000\du)--(30.845000\du,6.500000\du)--cycle;
\definecolor{dialinecolor}{rgb}{0.000000, 0.000000, 0.000000}
\pgfsetstrokecolor{dialinecolor}
\draw (24.960000\du,6.500000\du)--(24.960000\du,6.900000\du)--(30.845000\du,6.900000\du)--(30.845000\du,6.500000\du)--cycle;
\definecolor{dialinecolor}{rgb}{1.000000, 1.000000, 1.000000}
\pgfsetfillcolor{dialinecolor}
\fill (24.960000\du,6.900000\du)--(24.960000\du,7.300000\du)--(30.845000\du,7.300000\du)--(30.845000\du,6.900000\du)--cycle;
\definecolor{dialinecolor}{rgb}{0.000000, 0.000000, 0.000000}
\pgfsetstrokecolor{dialinecolor}
\draw (24.960000\du,6.900000\du)--(24.960000\du,7.300000\du)--(30.845000\du,7.300000\du)--(30.845000\du,6.900000\du)--cycle;
\pgfsetlinewidth{0.100000\du}
\pgfsetdash{}{0pt}
\definecolor{dialinecolor}{rgb}{1.000000, 1.000000, 1.000000}
\pgfsetfillcolor{dialinecolor}
\fill (15.990000\du,2.325000\du)--(15.990000\du,3.725000\du)--(20.557500\du,3.725000\du)--(20.557500\du,2.325000\du)--cycle;
\definecolor{dialinecolor}{rgb}{0.000000, 0.000000, 0.000000}
\pgfsetstrokecolor{dialinecolor}
\draw (15.990000\du,2.325000\du)--(15.990000\du,3.725000\du)--(20.557500\du,3.725000\du)--(20.557500\du,2.325000\du)--cycle;
% setfont left to latex
\definecolor{dialinecolor}{rgb}{0.000000, 0.000000, 0.000000}
\pgfsetstrokecolor{dialinecolor}
\node at (18.273750\du,3.275000\du){PersonUI};
\definecolor{dialinecolor}{rgb}{1.000000, 1.000000, 1.000000}
\pgfsetfillcolor{dialinecolor}
\fill (15.990000\du,3.725000\du)--(15.990000\du,4.125000\du)--(20.557500\du,4.125000\du)--(20.557500\du,3.725000\du)--cycle;
\definecolor{dialinecolor}{rgb}{0.000000, 0.000000, 0.000000}
\pgfsetstrokecolor{dialinecolor}
\draw (15.990000\du,3.725000\du)--(15.990000\du,4.125000\du)--(20.557500\du,4.125000\du)--(20.557500\du,3.725000\du)--cycle;
\definecolor{dialinecolor}{rgb}{1.000000, 1.000000, 1.000000}
\pgfsetfillcolor{dialinecolor}
\fill (15.990000\du,4.125000\du)--(15.990000\du,4.525000\du)--(20.557500\du,4.525000\du)--(20.557500\du,4.125000\du)--cycle;
\definecolor{dialinecolor}{rgb}{0.000000, 0.000000, 0.000000}
\pgfsetstrokecolor{dialinecolor}
\draw (15.990000\du,4.125000\du)--(15.990000\du,4.525000\du)--(20.557500\du,4.525000\du)--(20.557500\du,4.125000\du)--cycle;
\pgfsetlinewidth{0.100000\du}
\pgfsetdash{}{0pt}
\definecolor{dialinecolor}{rgb}{1.000000, 1.000000, 1.000000}
\pgfsetfillcolor{dialinecolor}
\fill (25.970000\du,8.050000\du)--(25.970000\du,9.450000\du)--(30.842500\du,9.450000\du)--(30.842500\du,8.050000\du)--cycle;
\definecolor{dialinecolor}{rgb}{0.000000, 0.000000, 0.000000}
\pgfsetstrokecolor{dialinecolor}
\draw (25.970000\du,8.050000\du)--(25.970000\du,9.450000\du)--(30.842500\du,9.450000\du)--(30.842500\du,8.050000\du)--cycle;
% setfont left to latex
\definecolor{dialinecolor}{rgb}{0.000000, 0.000000, 0.000000}
\pgfsetstrokecolor{dialinecolor}
\node at (28.406250\du,9.000000\du){ChartLine};
\definecolor{dialinecolor}{rgb}{1.000000, 1.000000, 1.000000}
\pgfsetfillcolor{dialinecolor}
\fill (25.970000\du,9.450000\du)--(25.970000\du,9.850000\du)--(30.842500\du,9.850000\du)--(30.842500\du,9.450000\du)--cycle;
\definecolor{dialinecolor}{rgb}{0.000000, 0.000000, 0.000000}
\pgfsetstrokecolor{dialinecolor}
\draw (25.970000\du,9.450000\du)--(25.970000\du,9.850000\du)--(30.842500\du,9.850000\du)--(30.842500\du,9.450000\du)--cycle;
\definecolor{dialinecolor}{rgb}{1.000000, 1.000000, 1.000000}
\pgfsetfillcolor{dialinecolor}
\fill (25.970000\du,9.850000\du)--(25.970000\du,10.250000\du)--(30.842500\du,10.250000\du)--(30.842500\du,9.850000\du)--cycle;
\definecolor{dialinecolor}{rgb}{0.000000, 0.000000, 0.000000}
\pgfsetstrokecolor{dialinecolor}
\draw (25.970000\du,9.850000\du)--(25.970000\du,10.250000\du)--(30.842500\du,10.250000\du)--(30.842500\du,9.850000\du)--cycle;
\pgfsetlinewidth{0.100000\du}
\pgfsetdash{}{0pt}
\definecolor{dialinecolor}{rgb}{1.000000, 1.000000, 1.000000}
\pgfsetfillcolor{dialinecolor}
\fill (24.750000\du,2.225000\du)--(24.750000\du,3.625000\du)--(31.020000\du,3.625000\du)--(31.020000\du,2.225000\du)--cycle;
\definecolor{dialinecolor}{rgb}{0.000000, 0.000000, 0.000000}
\pgfsetstrokecolor{dialinecolor}
\draw (24.750000\du,2.225000\du)--(24.750000\du,3.625000\du)--(31.020000\du,3.625000\du)--(31.020000\du,2.225000\du)--cycle;
% setfont left to latex
\definecolor{dialinecolor}{rgb}{0.000000, 0.000000, 0.000000}
\pgfsetstrokecolor{dialinecolor}
\node at (27.885000\du,3.175000\du){MainWindow};
\definecolor{dialinecolor}{rgb}{1.000000, 1.000000, 1.000000}
\pgfsetfillcolor{dialinecolor}
\fill (24.750000\du,3.625000\du)--(24.750000\du,4.025000\du)--(31.020000\du,4.025000\du)--(31.020000\du,3.625000\du)--cycle;
\definecolor{dialinecolor}{rgb}{0.000000, 0.000000, 0.000000}
\pgfsetstrokecolor{dialinecolor}
\draw (24.750000\du,3.625000\du)--(24.750000\du,4.025000\du)--(31.020000\du,4.025000\du)--(31.020000\du,3.625000\du)--cycle;
\definecolor{dialinecolor}{rgb}{1.000000, 1.000000, 1.000000}
\pgfsetfillcolor{dialinecolor}
\fill (24.750000\du,4.025000\du)--(24.750000\du,4.425000\du)--(31.020000\du,4.425000\du)--(31.020000\du,4.025000\du)--cycle;
\definecolor{dialinecolor}{rgb}{0.000000, 0.000000, 0.000000}
\pgfsetstrokecolor{dialinecolor}
\draw (24.750000\du,4.025000\du)--(24.750000\du,4.425000\du)--(31.020000\du,4.425000\du)--(31.020000\du,4.025000\du)--cycle;
\pgfsetlinewidth{0.100000\du}
\pgfsetdash{}{0pt}
\pgfsetdash{}{0pt}
\pgfsetbuttcap
{
\definecolor{dialinecolor}{rgb}{0.000000, 0.000000, 0.000000}
\pgfsetfillcolor{dialinecolor}
% was here!!!
\pgfsetarrowsstart{to}
\pgfsetarrowsend{to}
\definecolor{dialinecolor}{rgb}{0.000000, 0.000000, 0.000000}
\pgfsetstrokecolor{dialinecolor}
\draw (13.590000\du,3.050000\du)--(15.990000\du,3.025000\du);
}
\pgfsetlinewidth{0.100000\du}
\pgfsetdash{}{0pt}
\pgfsetdash{}{0pt}
\pgfsetbuttcap
{
\definecolor{dialinecolor}{rgb}{0.000000, 0.000000, 0.000000}
\pgfsetfillcolor{dialinecolor}
% was here!!!
\pgfsetarrowsstart{to}
\pgfsetarrowsend{to}
\definecolor{dialinecolor}{rgb}{0.000000, 0.000000, 0.000000}
\pgfsetstrokecolor{dialinecolor}
\draw (13.560000\du,8.550000\du)--(15.950000\du,8.500000\du);
}
\end{tikzpicture}

	
	\begin{figure}[!H]
	\centering
	% Graphic for TeX using PGF
% Title: /home/albin/Developpements/EclipseWorkspace/Automaton-Cellular/UML.dia
% Creator: Dia v0.97.2
% CreationDate: Fri May 18 21:42:13 2012
% For: albin
% \usepackage{tikz}
% The following commands are not supported in PSTricks at present
% We define them conditionally, so when they are implemented,
% this pgf file will use them.
\ifx\du\undefined
  \newlength{\du}
\fi
\setlength{\du}{15\unitlength}
\begin{tikzpicture}
\pgftransformxscale{1.000000}
\pgftransformyscale{-1.000000}
\definecolor{dialinecolor}{rgb}{0.000000, 0.000000, 0.000000}
\pgfsetstrokecolor{dialinecolor}
\definecolor{dialinecolor}{rgb}{1.000000, 1.000000, 1.000000}
\pgfsetfillcolor{dialinecolor}
\pgfsetlinewidth{0.100000\du}
\pgfsetdash{}{0pt}
\definecolor{dialinecolor}{rgb}{1.000000, 1.000000, 1.000000}
\pgfsetfillcolor{dialinecolor}
\fill (1.400000\du,1.900000\du)--(1.400000\du,10.800000\du)--(14.450000\du,10.800000\du)--(14.450000\du,1.900000\du)--cycle;
\definecolor{dialinecolor}{rgb}{0.000000, 0.000000, 0.000000}
\pgfsetstrokecolor{dialinecolor}
\draw (1.400000\du,1.900000\du)--(1.400000\du,10.800000\du)--(14.450000\du,10.800000\du)--(14.450000\du,1.900000\du)--cycle;
\definecolor{dialinecolor}{rgb}{1.000000, 1.000000, 1.000000}
\pgfsetfillcolor{dialinecolor}
\fill (1.400000\du,0.900000\du)--(1.400000\du,1.900000\du)--(3.525000\du,1.900000\du)--(3.525000\du,0.900000\du)--cycle;
\definecolor{dialinecolor}{rgb}{0.000000, 0.000000, 0.000000}
\pgfsetstrokecolor{dialinecolor}
\draw (1.400000\du,0.900000\du)--(1.400000\du,1.900000\du)--(3.525000\du,1.900000\du)--(3.525000\du,0.900000\du)--cycle;
% setfont left to latex
\definecolor{dialinecolor}{rgb}{0.000000, 0.000000, 0.000000}
\pgfsetstrokecolor{dialinecolor}
\node[anchor=west] at (1.500000\du,1.650000\du){Model};
\pgfsetlinewidth{0.100000\du}
\pgfsetdash{}{0pt}
\definecolor{dialinecolor}{rgb}{1.000000, 1.000000, 1.000000}
\pgfsetfillcolor{dialinecolor}
\fill (1.750000\du,2.400000\du)--(1.750000\du,3.800000\du)--(7.272500\du,3.800000\du)--(7.272500\du,2.400000\du)--cycle;
\definecolor{dialinecolor}{rgb}{0.000000, 0.000000, 0.000000}
\pgfsetstrokecolor{dialinecolor}
\draw (1.750000\du,2.400000\du)--(1.750000\du,3.800000\du)--(7.272500\du,3.800000\du)--(7.272500\du,2.400000\du)--cycle;
% setfont left to latex
\definecolor{dialinecolor}{rgb}{0.000000, 0.000000, 0.000000}
\pgfsetstrokecolor{dialinecolor}
\node at (4.511250\du,3.350000\du){MathModel};
\definecolor{dialinecolor}{rgb}{1.000000, 1.000000, 1.000000}
\pgfsetfillcolor{dialinecolor}
\fill (1.750000\du,3.800000\du)--(1.750000\du,4.200000\du)--(7.272500\du,4.200000\du)--(7.272500\du,3.800000\du)--cycle;
\definecolor{dialinecolor}{rgb}{0.000000, 0.000000, 0.000000}
\pgfsetstrokecolor{dialinecolor}
\draw (1.750000\du,3.800000\du)--(1.750000\du,4.200000\du)--(7.272500\du,4.200000\du)--(7.272500\du,3.800000\du)--cycle;
\definecolor{dialinecolor}{rgb}{1.000000, 1.000000, 1.000000}
\pgfsetfillcolor{dialinecolor}
\fill (1.750000\du,4.200000\du)--(1.750000\du,4.600000\du)--(7.272500\du,4.600000\du)--(7.272500\du,4.200000\du)--cycle;
\definecolor{dialinecolor}{rgb}{0.000000, 0.000000, 0.000000}
\pgfsetstrokecolor{dialinecolor}
\draw (1.750000\du,4.200000\du)--(1.750000\du,4.600000\du)--(7.272500\du,4.600000\du)--(7.272500\du,4.200000\du)--cycle;
\pgfsetlinewidth{0.100000\du}
\pgfsetdash{}{0pt}
\definecolor{dialinecolor}{rgb}{1.000000, 1.000000, 1.000000}
\pgfsetfillcolor{dialinecolor}
\fill (1.780000\du,5.025000\du)--(1.780000\du,6.425000\du)--(8.670000\du,6.425000\du)--(8.670000\du,5.025000\du)--cycle;
\definecolor{dialinecolor}{rgb}{0.000000, 0.000000, 0.000000}
\pgfsetstrokecolor{dialinecolor}
\draw (1.780000\du,5.025000\du)--(1.780000\du,6.425000\du)--(8.670000\du,6.425000\du)--(8.670000\du,5.025000\du)--cycle;
% setfont left to latex
\definecolor{dialinecolor}{rgb}{0.000000, 0.000000, 0.000000}
\pgfsetstrokecolor{dialinecolor}
\node at (5.225000\du,5.975000\du){Neighborhood};
\definecolor{dialinecolor}{rgb}{1.000000, 1.000000, 1.000000}
\pgfsetfillcolor{dialinecolor}
\fill (1.780000\du,6.425000\du)--(1.780000\du,6.825000\du)--(8.670000\du,6.825000\du)--(8.670000\du,6.425000\du)--cycle;
\definecolor{dialinecolor}{rgb}{0.000000, 0.000000, 0.000000}
\pgfsetstrokecolor{dialinecolor}
\draw (1.780000\du,6.425000\du)--(1.780000\du,6.825000\du)--(8.670000\du,6.825000\du)--(8.670000\du,6.425000\du)--cycle;
\definecolor{dialinecolor}{rgb}{1.000000, 1.000000, 1.000000}
\pgfsetfillcolor{dialinecolor}
\fill (1.780000\du,6.825000\du)--(1.780000\du,7.225000\du)--(8.670000\du,7.225000\du)--(8.670000\du,6.825000\du)--cycle;
\definecolor{dialinecolor}{rgb}{0.000000, 0.000000, 0.000000}
\pgfsetstrokecolor{dialinecolor}
\draw (1.780000\du,6.825000\du)--(1.780000\du,7.225000\du)--(8.670000\du,7.225000\du)--(8.670000\du,6.825000\du)--cycle;
\pgfsetlinewidth{0.100000\du}
\pgfsetdash{}{0pt}
\definecolor{dialinecolor}{rgb}{1.000000, 1.000000, 1.000000}
\pgfsetfillcolor{dialinecolor}
\fill (11.160000\du,7.850000\du)--(11.160000\du,9.250000\du)--(13.560000\du,9.250000\du)--(13.560000\du,7.850000\du)--cycle;
\definecolor{dialinecolor}{rgb}{0.000000, 0.000000, 0.000000}
\pgfsetstrokecolor{dialinecolor}
\draw (11.160000\du,7.850000\du)--(11.160000\du,9.250000\du)--(13.560000\du,9.250000\du)--(13.560000\du,7.850000\du)--cycle;
% setfont left to latex
\definecolor{dialinecolor}{rgb}{0.000000, 0.000000, 0.000000}
\pgfsetstrokecolor{dialinecolor}
\node at (12.360000\du,8.800000\du){Grid};
\definecolor{dialinecolor}{rgb}{1.000000, 1.000000, 1.000000}
\pgfsetfillcolor{dialinecolor}
\fill (11.160000\du,9.250000\du)--(11.160000\du,9.650000\du)--(13.560000\du,9.650000\du)--(13.560000\du,9.250000\du)--cycle;
\definecolor{dialinecolor}{rgb}{0.000000, 0.000000, 0.000000}
\pgfsetstrokecolor{dialinecolor}
\draw (11.160000\du,9.250000\du)--(11.160000\du,9.650000\du)--(13.560000\du,9.650000\du)--(13.560000\du,9.250000\du)--cycle;
\definecolor{dialinecolor}{rgb}{1.000000, 1.000000, 1.000000}
\pgfsetfillcolor{dialinecolor}
\fill (11.160000\du,9.650000\du)--(11.160000\du,10.050000\du)--(13.560000\du,10.050000\du)--(13.560000\du,9.650000\du)--cycle;
\definecolor{dialinecolor}{rgb}{0.000000, 0.000000, 0.000000}
\pgfsetstrokecolor{dialinecolor}
\draw (11.160000\du,9.650000\du)--(11.160000\du,10.050000\du)--(13.560000\du,10.050000\du)--(13.560000\du,9.650000\du)--cycle;
\pgfsetlinewidth{0.100000\du}
\pgfsetdash{}{0pt}
\definecolor{dialinecolor}{rgb}{1.000000, 1.000000, 1.000000}
\pgfsetfillcolor{dialinecolor}
\fill (1.790000\du,7.725000\du)--(1.790000\du,9.125000\du)--(6.232500\du,9.125000\du)--(6.232500\du,7.725000\du)--cycle;
\definecolor{dialinecolor}{rgb}{0.000000, 0.000000, 0.000000}
\pgfsetstrokecolor{dialinecolor}
\draw (1.790000\du,7.725000\du)--(1.790000\du,9.125000\du)--(6.232500\du,9.125000\du)--(6.232500\du,7.725000\du)--cycle;
% setfont left to latex
\definecolor{dialinecolor}{rgb}{0.000000, 0.000000, 0.000000}
\pgfsetstrokecolor{dialinecolor}
\node at (4.011250\du,8.675000\du){Obstacle};
\definecolor{dialinecolor}{rgb}{1.000000, 1.000000, 1.000000}
\pgfsetfillcolor{dialinecolor}
\fill (1.790000\du,9.125000\du)--(1.790000\du,9.525000\du)--(6.232500\du,9.525000\du)--(6.232500\du,9.125000\du)--cycle;
\definecolor{dialinecolor}{rgb}{0.000000, 0.000000, 0.000000}
\pgfsetstrokecolor{dialinecolor}
\draw (1.790000\du,9.125000\du)--(1.790000\du,9.525000\du)--(6.232500\du,9.525000\du)--(6.232500\du,9.125000\du)--cycle;
\definecolor{dialinecolor}{rgb}{1.000000, 1.000000, 1.000000}
\pgfsetfillcolor{dialinecolor}
\fill (1.790000\du,9.525000\du)--(1.790000\du,9.925000\du)--(6.232500\du,9.925000\du)--(6.232500\du,9.525000\du)--cycle;
\definecolor{dialinecolor}{rgb}{0.000000, 0.000000, 0.000000}
\pgfsetstrokecolor{dialinecolor}
\draw (1.790000\du,9.525000\du)--(1.790000\du,9.925000\du)--(6.232500\du,9.925000\du)--(6.232500\du,9.525000\du)--cycle;
\pgfsetlinewidth{0.100000\du}
\pgfsetdash{}{0pt}
\definecolor{dialinecolor}{rgb}{1.000000, 1.000000, 1.000000}
\pgfsetfillcolor{dialinecolor}
\fill (9.970000\du,2.350000\du)--(9.970000\du,3.750000\du)--(13.590000\du,3.750000\du)--(13.590000\du,2.350000\du)--cycle;
\definecolor{dialinecolor}{rgb}{0.000000, 0.000000, 0.000000}
\pgfsetstrokecolor{dialinecolor}
\draw (9.970000\du,2.350000\du)--(9.970000\du,3.750000\du)--(13.590000\du,3.750000\du)--(13.590000\du,2.350000\du)--cycle;
% setfont left to latex
\definecolor{dialinecolor}{rgb}{0.000000, 0.000000, 0.000000}
\pgfsetstrokecolor{dialinecolor}
\node at (11.780000\du,3.300000\du){Person};
\definecolor{dialinecolor}{rgb}{1.000000, 1.000000, 1.000000}
\pgfsetfillcolor{dialinecolor}
\fill (9.970000\du,3.750000\du)--(9.970000\du,4.150000\du)--(13.590000\du,4.150000\du)--(13.590000\du,3.750000\du)--cycle;
\definecolor{dialinecolor}{rgb}{0.000000, 0.000000, 0.000000}
\pgfsetstrokecolor{dialinecolor}
\draw (9.970000\du,3.750000\du)--(9.970000\du,4.150000\du)--(13.590000\du,4.150000\du)--(13.590000\du,3.750000\du)--cycle;
\definecolor{dialinecolor}{rgb}{1.000000, 1.000000, 1.000000}
\pgfsetfillcolor{dialinecolor}
\fill (9.970000\du,4.150000\du)--(9.970000\du,4.550000\du)--(13.590000\du,4.550000\du)--(13.590000\du,4.150000\du)--cycle;
\definecolor{dialinecolor}{rgb}{0.000000, 0.000000, 0.000000}
\pgfsetstrokecolor{dialinecolor}
\draw (9.970000\du,4.150000\du)--(9.970000\du,4.550000\du)--(13.590000\du,4.550000\du)--(13.590000\du,4.150000\du)--cycle;
\pgfsetlinewidth{0.100000\du}
\pgfsetdash{}{0pt}
\definecolor{dialinecolor}{rgb}{1.000000, 1.000000, 1.000000}
\pgfsetfillcolor{dialinecolor}
\fill (1.450000\du,12.650000\du)--(1.450000\du,16.150000\du)--(9.700000\du,16.150000\du)--(9.700000\du,12.650000\du)--cycle;
\definecolor{dialinecolor}{rgb}{0.000000, 0.000000, 0.000000}
\pgfsetstrokecolor{dialinecolor}
\draw (1.450000\du,12.650000\du)--(1.450000\du,16.150000\du)--(9.700000\du,16.150000\du)--(9.700000\du,12.650000\du)--cycle;
\definecolor{dialinecolor}{rgb}{1.000000, 1.000000, 1.000000}
\pgfsetfillcolor{dialinecolor}
\fill (1.450000\du,11.650000\du)--(1.450000\du,12.650000\du)--(5.500000\du,12.650000\du)--(5.500000\du,11.650000\du)--cycle;
\definecolor{dialinecolor}{rgb}{0.000000, 0.000000, 0.000000}
\pgfsetstrokecolor{dialinecolor}
\draw (1.450000\du,11.650000\du)--(1.450000\du,12.650000\du)--(5.500000\du,12.650000\du)--(5.500000\du,11.650000\du)--cycle;
% setfont left to latex
\definecolor{dialinecolor}{rgb}{0.000000, 0.000000, 0.000000}
\pgfsetstrokecolor{dialinecolor}
\node[anchor=west] at (1.550000\du,12.400000\du){Controller};
\pgfsetlinewidth{0.100000\du}
\pgfsetdash{}{0pt}
\definecolor{dialinecolor}{rgb}{1.000000, 1.000000, 1.000000}
\pgfsetfillcolor{dialinecolor}
\fill (3.100000\du,13.350000\du)--(3.100000\du,14.750000\du)--(8.122500\du,14.750000\du)--(8.122500\du,13.350000\du)--cycle;
\definecolor{dialinecolor}{rgb}{0.000000, 0.000000, 0.000000}
\pgfsetstrokecolor{dialinecolor}
\draw (3.100000\du,13.350000\du)--(3.100000\du,14.750000\du)--(8.122500\du,14.750000\du)--(8.122500\du,13.350000\du)--cycle;
% setfont left to latex
\definecolor{dialinecolor}{rgb}{0.000000, 0.000000, 0.000000}
\pgfsetstrokecolor{dialinecolor}
\node at (5.611250\du,14.300000\du){Controller};
\definecolor{dialinecolor}{rgb}{1.000000, 1.000000, 1.000000}
\pgfsetfillcolor{dialinecolor}
\fill (3.100000\du,14.750000\du)--(3.100000\du,15.150000\du)--(8.122500\du,15.150000\du)--(8.122500\du,14.750000\du)--cycle;
\definecolor{dialinecolor}{rgb}{0.000000, 0.000000, 0.000000}
\pgfsetstrokecolor{dialinecolor}
\draw (3.100000\du,14.750000\du)--(3.100000\du,15.150000\du)--(8.122500\du,15.150000\du)--(8.122500\du,14.750000\du)--cycle;
\definecolor{dialinecolor}{rgb}{1.000000, 1.000000, 1.000000}
\pgfsetfillcolor{dialinecolor}
\fill (3.100000\du,15.150000\du)--(3.100000\du,15.550000\du)--(8.122500\du,15.550000\du)--(8.122500\du,15.150000\du)--cycle;
\definecolor{dialinecolor}{rgb}{0.000000, 0.000000, 0.000000}
\pgfsetstrokecolor{dialinecolor}
\draw (3.100000\du,15.150000\du)--(3.100000\du,15.550000\du)--(8.122500\du,15.550000\du)--(8.122500\du,15.150000\du)--cycle;
\pgfsetlinewidth{0.100000\du}
\pgfsetdash{}{0pt}
\definecolor{dialinecolor}{rgb}{1.000000, 1.000000, 1.000000}
\pgfsetfillcolor{dialinecolor}
\fill (15.350000\du,1.750000\du)--(15.350000\du,10.850000\du)--(31.600000\du,10.850000\du)--(31.600000\du,1.750000\du)--cycle;
\definecolor{dialinecolor}{rgb}{0.000000, 0.000000, 0.000000}
\pgfsetstrokecolor{dialinecolor}
\draw (15.350000\du,1.750000\du)--(15.350000\du,10.850000\du)--(31.600000\du,10.850000\du)--(31.600000\du,1.750000\du)--cycle;
\definecolor{dialinecolor}{rgb}{1.000000, 1.000000, 1.000000}
\pgfsetfillcolor{dialinecolor}
\fill (15.350000\du,0.750000\du)--(15.350000\du,1.750000\du)--(17.350000\du,1.750000\du)--(17.350000\du,0.750000\du)--cycle;
\definecolor{dialinecolor}{rgb}{0.000000, 0.000000, 0.000000}
\pgfsetstrokecolor{dialinecolor}
\draw (15.350000\du,0.750000\du)--(15.350000\du,1.750000\du)--(17.350000\du,1.750000\du)--(17.350000\du,0.750000\du)--cycle;
% setfont left to latex
\definecolor{dialinecolor}{rgb}{0.000000, 0.000000, 0.000000}
\pgfsetstrokecolor{dialinecolor}
\node[anchor=west] at (15.450000\du,1.500000\du){View};
\pgfsetlinewidth{0.100000\du}
\pgfsetdash{}{0pt}
\definecolor{dialinecolor}{rgb}{1.000000, 1.000000, 1.000000}
\pgfsetfillcolor{dialinecolor}
\fill (15.950000\du,7.800000\du)--(15.950000\du,9.200000\du)--(19.297500\du,9.200000\du)--(19.297500\du,7.800000\du)--cycle;
\definecolor{dialinecolor}{rgb}{0.000000, 0.000000, 0.000000}
\pgfsetstrokecolor{dialinecolor}
\draw (15.950000\du,7.800000\du)--(15.950000\du,9.200000\du)--(19.297500\du,9.200000\du)--(19.297500\du,7.800000\du)--cycle;
% setfont left to latex
\definecolor{dialinecolor}{rgb}{0.000000, 0.000000, 0.000000}
\pgfsetstrokecolor{dialinecolor}
\node at (17.623750\du,8.750000\du){GridUI};
\definecolor{dialinecolor}{rgb}{1.000000, 1.000000, 1.000000}
\pgfsetfillcolor{dialinecolor}
\fill (15.950000\du,9.200000\du)--(15.950000\du,9.600000\du)--(19.297500\du,9.600000\du)--(19.297500\du,9.200000\du)--cycle;
\definecolor{dialinecolor}{rgb}{0.000000, 0.000000, 0.000000}
\pgfsetstrokecolor{dialinecolor}
\draw (15.950000\du,9.200000\du)--(15.950000\du,9.600000\du)--(19.297500\du,9.600000\du)--(19.297500\du,9.200000\du)--cycle;
\definecolor{dialinecolor}{rgb}{1.000000, 1.000000, 1.000000}
\pgfsetfillcolor{dialinecolor}
\fill (15.950000\du,9.600000\du)--(15.950000\du,10.000000\du)--(19.297500\du,10.000000\du)--(19.297500\du,9.600000\du)--cycle;
\definecolor{dialinecolor}{rgb}{0.000000, 0.000000, 0.000000}
\pgfsetstrokecolor{dialinecolor}
\draw (15.950000\du,9.600000\du)--(15.950000\du,10.000000\du)--(19.297500\du,10.000000\du)--(19.297500\du,9.600000\du)--cycle;
\pgfsetlinewidth{0.100000\du}
\pgfsetdash{}{0pt}
\definecolor{dialinecolor}{rgb}{1.000000, 1.000000, 1.000000}
\pgfsetfillcolor{dialinecolor}
\fill (21.780000\du,8.125000\du)--(21.780000\du,9.525000\du)--(24.755000\du,9.525000\du)--(24.755000\du,8.125000\du)--cycle;
\definecolor{dialinecolor}{rgb}{0.000000, 0.000000, 0.000000}
\pgfsetstrokecolor{dialinecolor}
\draw (21.780000\du,8.125000\du)--(21.780000\du,9.525000\du)--(24.755000\du,9.525000\du)--(24.755000\du,8.125000\du)--cycle;
% setfont left to latex
\definecolor{dialinecolor}{rgb}{0.000000, 0.000000, 0.000000}
\pgfsetstrokecolor{dialinecolor}
\node at (23.267500\du,9.075000\du){Chart};
\definecolor{dialinecolor}{rgb}{1.000000, 1.000000, 1.000000}
\pgfsetfillcolor{dialinecolor}
\fill (21.780000\du,9.525000\du)--(21.780000\du,9.925000\du)--(24.755000\du,9.925000\du)--(24.755000\du,9.525000\du)--cycle;
\definecolor{dialinecolor}{rgb}{0.000000, 0.000000, 0.000000}
\pgfsetstrokecolor{dialinecolor}
\draw (21.780000\du,9.525000\du)--(21.780000\du,9.925000\du)--(24.755000\du,9.925000\du)--(24.755000\du,9.525000\du)--cycle;
\definecolor{dialinecolor}{rgb}{1.000000, 1.000000, 1.000000}
\pgfsetfillcolor{dialinecolor}
\fill (21.780000\du,9.925000\du)--(21.780000\du,10.325000\du)--(24.755000\du,10.325000\du)--(24.755000\du,9.925000\du)--cycle;
\definecolor{dialinecolor}{rgb}{0.000000, 0.000000, 0.000000}
\pgfsetstrokecolor{dialinecolor}
\draw (21.780000\du,9.925000\du)--(21.780000\du,10.325000\du)--(24.755000\du,10.325000\du)--(24.755000\du,9.925000\du)--cycle;
\pgfsetlinewidth{0.100000\du}
\pgfsetdash{}{0pt}
\definecolor{dialinecolor}{rgb}{1.000000, 1.000000, 1.000000}
\pgfsetfillcolor{dialinecolor}
\fill (24.960000\du,5.100000\du)--(24.960000\du,6.500000\du)--(30.845000\du,6.500000\du)--(30.845000\du,5.100000\du)--cycle;
\definecolor{dialinecolor}{rgb}{0.000000, 0.000000, 0.000000}
\pgfsetstrokecolor{dialinecolor}
\draw (24.960000\du,5.100000\du)--(24.960000\du,6.500000\du)--(30.845000\du,6.500000\du)--(30.845000\du,5.100000\du)--cycle;
% setfont left to latex
\definecolor{dialinecolor}{rgb}{0.000000, 0.000000, 0.000000}
\pgfsetstrokecolor{dialinecolor}
\node at (27.902500\du,6.050000\du){DrawAreaUI};
\definecolor{dialinecolor}{rgb}{1.000000, 1.000000, 1.000000}
\pgfsetfillcolor{dialinecolor}
\fill (24.960000\du,6.500000\du)--(24.960000\du,6.900000\du)--(30.845000\du,6.900000\du)--(30.845000\du,6.500000\du)--cycle;
\definecolor{dialinecolor}{rgb}{0.000000, 0.000000, 0.000000}
\pgfsetstrokecolor{dialinecolor}
\draw (24.960000\du,6.500000\du)--(24.960000\du,6.900000\du)--(30.845000\du,6.900000\du)--(30.845000\du,6.500000\du)--cycle;
\definecolor{dialinecolor}{rgb}{1.000000, 1.000000, 1.000000}
\pgfsetfillcolor{dialinecolor}
\fill (24.960000\du,6.900000\du)--(24.960000\du,7.300000\du)--(30.845000\du,7.300000\du)--(30.845000\du,6.900000\du)--cycle;
\definecolor{dialinecolor}{rgb}{0.000000, 0.000000, 0.000000}
\pgfsetstrokecolor{dialinecolor}
\draw (24.960000\du,6.900000\du)--(24.960000\du,7.300000\du)--(30.845000\du,7.300000\du)--(30.845000\du,6.900000\du)--cycle;
\pgfsetlinewidth{0.100000\du}
\pgfsetdash{}{0pt}
\definecolor{dialinecolor}{rgb}{1.000000, 1.000000, 1.000000}
\pgfsetfillcolor{dialinecolor}
\fill (15.990000\du,2.325000\du)--(15.990000\du,3.725000\du)--(20.557500\du,3.725000\du)--(20.557500\du,2.325000\du)--cycle;
\definecolor{dialinecolor}{rgb}{0.000000, 0.000000, 0.000000}
\pgfsetstrokecolor{dialinecolor}
\draw (15.990000\du,2.325000\du)--(15.990000\du,3.725000\du)--(20.557500\du,3.725000\du)--(20.557500\du,2.325000\du)--cycle;
% setfont left to latex
\definecolor{dialinecolor}{rgb}{0.000000, 0.000000, 0.000000}
\pgfsetstrokecolor{dialinecolor}
\node at (18.273750\du,3.275000\du){PersonUI};
\definecolor{dialinecolor}{rgb}{1.000000, 1.000000, 1.000000}
\pgfsetfillcolor{dialinecolor}
\fill (15.990000\du,3.725000\du)--(15.990000\du,4.125000\du)--(20.557500\du,4.125000\du)--(20.557500\du,3.725000\du)--cycle;
\definecolor{dialinecolor}{rgb}{0.000000, 0.000000, 0.000000}
\pgfsetstrokecolor{dialinecolor}
\draw (15.990000\du,3.725000\du)--(15.990000\du,4.125000\du)--(20.557500\du,4.125000\du)--(20.557500\du,3.725000\du)--cycle;
\definecolor{dialinecolor}{rgb}{1.000000, 1.000000, 1.000000}
\pgfsetfillcolor{dialinecolor}
\fill (15.990000\du,4.125000\du)--(15.990000\du,4.525000\du)--(20.557500\du,4.525000\du)--(20.557500\du,4.125000\du)--cycle;
\definecolor{dialinecolor}{rgb}{0.000000, 0.000000, 0.000000}
\pgfsetstrokecolor{dialinecolor}
\draw (15.990000\du,4.125000\du)--(15.990000\du,4.525000\du)--(20.557500\du,4.525000\du)--(20.557500\du,4.125000\du)--cycle;
\pgfsetlinewidth{0.100000\du}
\pgfsetdash{}{0pt}
\definecolor{dialinecolor}{rgb}{1.000000, 1.000000, 1.000000}
\pgfsetfillcolor{dialinecolor}
\fill (25.970000\du,8.050000\du)--(25.970000\du,9.450000\du)--(30.842500\du,9.450000\du)--(30.842500\du,8.050000\du)--cycle;
\definecolor{dialinecolor}{rgb}{0.000000, 0.000000, 0.000000}
\pgfsetstrokecolor{dialinecolor}
\draw (25.970000\du,8.050000\du)--(25.970000\du,9.450000\du)--(30.842500\du,9.450000\du)--(30.842500\du,8.050000\du)--cycle;
% setfont left to latex
\definecolor{dialinecolor}{rgb}{0.000000, 0.000000, 0.000000}
\pgfsetstrokecolor{dialinecolor}
\node at (28.406250\du,9.000000\du){ChartLine};
\definecolor{dialinecolor}{rgb}{1.000000, 1.000000, 1.000000}
\pgfsetfillcolor{dialinecolor}
\fill (25.970000\du,9.450000\du)--(25.970000\du,9.850000\du)--(30.842500\du,9.850000\du)--(30.842500\du,9.450000\du)--cycle;
\definecolor{dialinecolor}{rgb}{0.000000, 0.000000, 0.000000}
\pgfsetstrokecolor{dialinecolor}
\draw (25.970000\du,9.450000\du)--(25.970000\du,9.850000\du)--(30.842500\du,9.850000\du)--(30.842500\du,9.450000\du)--cycle;
\definecolor{dialinecolor}{rgb}{1.000000, 1.000000, 1.000000}
\pgfsetfillcolor{dialinecolor}
\fill (25.970000\du,9.850000\du)--(25.970000\du,10.250000\du)--(30.842500\du,10.250000\du)--(30.842500\du,9.850000\du)--cycle;
\definecolor{dialinecolor}{rgb}{0.000000, 0.000000, 0.000000}
\pgfsetstrokecolor{dialinecolor}
\draw (25.970000\du,9.850000\du)--(25.970000\du,10.250000\du)--(30.842500\du,10.250000\du)--(30.842500\du,9.850000\du)--cycle;
\pgfsetlinewidth{0.100000\du}
\pgfsetdash{}{0pt}
\definecolor{dialinecolor}{rgb}{1.000000, 1.000000, 1.000000}
\pgfsetfillcolor{dialinecolor}
\fill (24.750000\du,2.225000\du)--(24.750000\du,3.625000\du)--(31.020000\du,3.625000\du)--(31.020000\du,2.225000\du)--cycle;
\definecolor{dialinecolor}{rgb}{0.000000, 0.000000, 0.000000}
\pgfsetstrokecolor{dialinecolor}
\draw (24.750000\du,2.225000\du)--(24.750000\du,3.625000\du)--(31.020000\du,3.625000\du)--(31.020000\du,2.225000\du)--cycle;
% setfont left to latex
\definecolor{dialinecolor}{rgb}{0.000000, 0.000000, 0.000000}
\pgfsetstrokecolor{dialinecolor}
\node at (27.885000\du,3.175000\du){MainWindow};
\definecolor{dialinecolor}{rgb}{1.000000, 1.000000, 1.000000}
\pgfsetfillcolor{dialinecolor}
\fill (24.750000\du,3.625000\du)--(24.750000\du,4.025000\du)--(31.020000\du,4.025000\du)--(31.020000\du,3.625000\du)--cycle;
\definecolor{dialinecolor}{rgb}{0.000000, 0.000000, 0.000000}
\pgfsetstrokecolor{dialinecolor}
\draw (24.750000\du,3.625000\du)--(24.750000\du,4.025000\du)--(31.020000\du,4.025000\du)--(31.020000\du,3.625000\du)--cycle;
\definecolor{dialinecolor}{rgb}{1.000000, 1.000000, 1.000000}
\pgfsetfillcolor{dialinecolor}
\fill (24.750000\du,4.025000\du)--(24.750000\du,4.425000\du)--(31.020000\du,4.425000\du)--(31.020000\du,4.025000\du)--cycle;
\definecolor{dialinecolor}{rgb}{0.000000, 0.000000, 0.000000}
\pgfsetstrokecolor{dialinecolor}
\draw (24.750000\du,4.025000\du)--(24.750000\du,4.425000\du)--(31.020000\du,4.425000\du)--(31.020000\du,4.025000\du)--cycle;
\pgfsetlinewidth{0.100000\du}
\pgfsetdash{}{0pt}
\pgfsetdash{}{0pt}
\pgfsetbuttcap
{
\definecolor{dialinecolor}{rgb}{0.000000, 0.000000, 0.000000}
\pgfsetfillcolor{dialinecolor}
% was here!!!
\pgfsetarrowsstart{to}
\pgfsetarrowsend{to}
\definecolor{dialinecolor}{rgb}{0.000000, 0.000000, 0.000000}
\pgfsetstrokecolor{dialinecolor}
\draw (13.590000\du,3.050000\du)--(15.990000\du,3.025000\du);
}
\pgfsetlinewidth{0.100000\du}
\pgfsetdash{}{0pt}
\pgfsetdash{}{0pt}
\pgfsetbuttcap
{
\definecolor{dialinecolor}{rgb}{0.000000, 0.000000, 0.000000}
\pgfsetfillcolor{dialinecolor}
% was here!!!
\pgfsetarrowsstart{to}
\pgfsetarrowsend{to}
\definecolor{dialinecolor}{rgb}{0.000000, 0.000000, 0.000000}
\pgfsetstrokecolor{dialinecolor}
\draw (13.560000\du,8.550000\du)--(15.950000\du,8.500000\du);
}
\end{tikzpicture}

	\caption{Schéma UML des classes}
	\label{fig:uml}
	\end{figure}
	
	