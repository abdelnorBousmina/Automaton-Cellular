\section{Modèle mathématique}
	Le modèle mathématique utilisé a été présenté dans un papier écrit par A. Varasa, M.D. Cornejoa, D. Mainemera, B. Toledob, J. Rogana, V. Munoza et J.A. Valdiviaa. \\
	Ce modèle représente une pièce comme une grille 2D. Chaque case de cette grille pouvant être vide, occupée par une personne ou par un obstacle. On considère que chaque case représente un espace de $0,4 * 0,4 m^2$ dans le monde réel. Cette surface est celle typiquement occupée par une personne dans une foule dense. Ainsi, en considérant une vitesse de marche moyenne de $1m/s$, lorsqu'une personne bouge de $0,4m$, on a donc $\Delta t \approx 0.4$.\\
	Ensuite, les mouvements des personnes sont déterminé par (1) un plan fixe de la configuration de la salle, calculé de façon à ce que la case la plus proche de la sortie soit préférée, et par (2) l'interaction entre les personnes.
	
	\subsection{La configuration de la salle}
	A la création du système, la taille de la pièce ainsi que la position de la sortie sont déterminées. En sachant cela, une valeur est assignée à chaque case, représentant sa distance à la porte : au plus la valeur est faible, au plus la case est proche de la sortie. C'est à partir de cette configuration statique que les personnes auront la possibilité de déterminer la prochaine case à rejoindre : celle qui a une valeur plus faible que là où il se trouve actuellement. \\
	L'algorithme utilisé est très clairement décrit dans le papier :
	\begin{enumerate}
	\item La pièce est divisée en une grille rectangulaire. La porte de sortie a une valeur de 1.
	\item Ensuite, toutes les cellules adjacente à la précédente (un second ``layer'' de cellules) se voient assigner des valeurs de la façon suivante :
		\begin{enumerate}
			\item Si une cellule a une valeur $N$, alors la cellule adjacente dans les directions verticales et horizontales ont une valeur $N + 1$. Nous allons permettre les déplacements en diagonale, donc il est pertinent de considérer les cellules en diagonale. On leur assigne alors une valeur $N + \lambda$ avec $\lambda \sup 1$. Nous faisons cela dans le but de représenter le fait que la distance entre deux cases diagonalement est plus grande que celle entre deux cases horizontalement ou verticalement adjacentes. (Le papier a choisi une valeur arbitraire de $\lambda = 3$. Dans notre projet, la valeur est personalisable).
			\item Si il y a un conflit lors d'une assignation de valeur, alors la valeur minimum est assignée.
		\end{enumerate}

		\item Puis le troisième ``layer'' de cellules est calculé, composé de toutes les cellules adjacentes à celle du second, mais pas celle du premier.
		\item Le processus est répété jusqu'à ce que toutes les cellules soient évaluées.
		\item Les murs sont également considérés lors de la définition de la grille. Les cellules représentant les murs ont des valeurs très élevées. Cela assurera que les personnes ne choisiront jamais de se placer sur ces cellules. (Les obstacles seront donc gérés de la même façon).
	\end{enumerate} 
